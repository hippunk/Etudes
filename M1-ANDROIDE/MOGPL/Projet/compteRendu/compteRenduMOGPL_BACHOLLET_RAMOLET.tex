%%%%%%%%%%%%%%%%%%%%%%%%%%%%%%%%%%%%%%%%%
% Short Sectioned Assignment
% LaTeX Template
% Version 1.0 (5/5/12)
%
% This template has been downloaded from:
% http://www.LaTeXTemplates.com
%
% Original author:
% Frits Wenneker (http://www.howtotex.com)
%
% License:
% CC BY-NC-SA 3.0 (http://creativecommons.org/licenses/by-nc-sa/3.0/)
%
%%%%%%%%%%%%%%%%%%%%%%%%%%%%%%%%%%%%%%%%%

%----------------------------------------------------------------------------------------
%	PACKAGES AND OTHER DOCUMENT CONFIGURATIONS
%----------------------------------------------------------------------------------------

\documentclass[paper=a4, fontsize=11pt]{scrartcl} % A4 paper and 11pt font size

\usepackage[francais]{babel}
\usepackage[T1]{fontenc} % Use 8-bit encoding that has 256 glyphs
%\usepackage{fourier} % Use the Adobe Utopia font for the document - comment this line to return to the LaTeX default
\usepackage{amsmath,amsfonts,amsthm} % Math packages

\usepackage{lipsum} % Used for inserting dummy 'Lorem ipsum' text into the template

\usepackage{sectsty} % Allows customizing section commands
\allsectionsfont{\normalfont\scshape} % Make all sections centered, the default font and small caps

\usepackage[usenames,dvipsnames]{color} % Required for custom colors
\usepackage{graphicx} % Required to insert images
\usepackage{listings} % Required for insertion of code

\usepackage{fancyhdr} % Custom headers and footers
\pagestyle{fancyplain} % Makes all pages in the document conform to the custom headers and footers
\fancyhead{} % No page header - if you want one, create it in the same way as the footers below
\fancyfoot[L]{} % Empty left footer
\fancyfoot[C]{} % Empty center footer
\fancyfoot[R]{\thepage} % Page numbering for right footer
\renewcommand{\headrulewidth}{0pt} % Remove header underlines
\renewcommand{\footrulewidth}{0pt} % Remove footer underlines
\setlength{\headheight}{13.6pt} % Customize the height of the header

\numberwithin{equation}{section} % Number equations within sections (i.e. 1.1, 1.2, 2.1, 2.2 instead of 1, 2, 3, 4)
\numberwithin{figure}{section} % Number figures within sections (i.e. 1.1, 1.2, 2.1, 2.2 instead of 1, 2, 3, 4)
\numberwithin{table}{section} % Number tables within sections (i.e. 1.1, 1.2, 2.1, 2.2 instead of 1, 2, 3, 4)

\setlength\parindent{0pt} % Removes all indentation from paragraphs - comment this line for an assignment with lots of text

%----------------------------------------------------------------------------------------
%	TITLE SECTION
%----------------------------------------------------------------------------------------

\newcommand{\horrule}[1]{\rule{\linewidth}{#1}} % Create horizontal rule command with 1 argument of height

\title{	
\normalfont \normalsize 
\textsc{Universit\'e Paris 6, Master ANDROIDE} \\ [25pt] % Your university, school and/or department name(s)
\horrule{0.5pt} \\[0.4cm] % Thin top horizontal rule
\huge Projet MOGPL \\ % The assignment title
\horrule{2pt} \\[0.5cm] % Thick bottom horizontal rule
}

\author{Olivier Bachollet, Arthur Ramolot} % Your name

\date{\normalsize\today} % Today's date or a custom date

\begin{document}

\maketitle % Print the title

%----------------------------------------------------------------------------------------
%	CODE INCLUSION CONFIGURATION
%----------------------------------------------------------------------------------------

\definecolor{MyDarkGreen}{rgb}{0.0,0.4,0.0} % This is the color used for comments
\lstloadlanguages{Python} % Load Perl syntax for listings, for a list of other languages supported see: ftp://ftp.tex.ac.uk/tex-archive/macros/latex/contrib/listings/listings.pdf
\lstset{language=Python, % Use Perl in this example
        frame=single, % Single frame around code
        basicstyle=\small\ttfamily, % Use small true type font
        keywordstyle=[1]\color{Blue}\bf, % Perl functions bold and blue
        keywordstyle=[2]\color{Purple}, % Perl function arguments purple
        keywordstyle=[3]\color{Blue}\underbar, % Custom functions underlined and blue
        identifierstyle=, % Nothing special about identifiers                                         
        commentstyle=\usefont{T1}{pcr}{m}{sl}\color{MyDarkGreen}\small, % Comments small dark green courier font
        stringstyle=\color{Purple}, % Strings are purple
        showstringspaces=false, % Don't put marks in string spaces
        tabsize=5, % 5 spaces per tab
        %
        % Put standard Perl functions not included in the default language here
        morekeywords={rand},
        %
        % Put Perl function parameters here
        morekeywords=[2]{on, off, interp},
        %
        % Put user defined functions here
        morekeywords=[3]{test},
       	%
        morecomment=[l][\color{Blue}]{...}, % Line continuation (...) like blue comment
        numbers=left, % Line numbers on left
        firstnumber=1, % Line numbers start with line 1
        numberstyle=\tiny\color{Blue}, % Line numbers are blue and small
        stepnumber=5 % Line numbers go in steps of 5
}

% Creates a new command to include a perl script, the first parameter is the filename of the script (without .pl), the second parameter is the caption
\newcommand{\cscript}[2]{
\begin{itemize}
\item[]\lstinputlisting[caption=#2,label=#1]{#1.py}
\end{itemize}
}

%----------------------------------------------------------------------------------------
%	TABLE OF CONTENTS
%----------------------------------------------------------------------------------------

%\setcounter{tocdepth}{1} % Uncomment this line if you don't want subsections listed in the ToC

\newpage
\tableofcontents
\newpage


%----------------------------------------------------------------------------------------
%	Premi\`ere mod\'elisation du probl\`eme
%----------------------------------------------------------------------------------------

\newpage
\section{Premi\`ere mod\'elisation du probl\`eme}

\subsection{Mod\'elisation du probl\`eme}

Fonction d'optimisation : \\\\
$\max\sum_{i=1}^{n}\sum_{j=1}^{m} u_{ij}.x_{ij}$\\\\
Contraintes :\\\\
$\prod_{i=1}^{n}\sum_{j=1}^{m}x_{ij} = 1$\\
$\prod_{j=1}^{n}\sum_{i=1}^{m}x_{ij} = 1$\\
$x_{ij}\in\{0,1\}, \forall i,j\in[0,n]$\\
$u_{ij}\in[0,M]$

\cscript{P0}{Programme lin\'eaire P0 avec pygurobi}

\subsection{Essais num\'eriques}

\begin{tabular}{|*{5}{c|}}
    \hline
     n  & t  & Moyenne  & Minimum  & Maximum\\
    \hline
     10  & 0.01  & 7.68  & 5.3  & 9.1\\
    \hline
     50  & 0.18  & 8.90  & 7.4 & 10\\
    \hline
     100  & 0.56  & 9.26 & 8.0 & 10\\
    \hline
     500  & 12.73 & 9.80 & 9.0 & 10\\
    \hline
     1000  & 69.79 & 9.98 & 9.0 & 10\\
    \hline
\end{tabular}

Comme attendu, plus n est grand plus le calcul est long.
Augmenter M n'influe pas sur les temps de calcul. 
Les autres donn\'ees \'evoluent proportionellement avec la valeur maximale des objets. 

%----------------------------------------------------------------------------------------
%	Approche \'egalitariste
%----------------------------------------------------------------------------------------

\newpage
\section{Approche \'egalitariste}

\subsection{L'approche flot max}

Pour \'ecrire le probl\`eme de l'existence d'une affectation dans laquelle les satisfactions des agents seraient toute sup\'erieures ou \'egales \`a lambda comme un probl\`eme de flot maximum dans un graphe il suffit de cr\'e\'e un graphe avec des sommets pour chaque objet et pour chaque agent, une source connect\'e \`a tous les agents, un puit connect\'e \`a tous les objet, et des arcs qui connecte les agents vers les objets qui leur donne une satisfaction sup\'erieure ou \'egale \`a lambda. Chaque arcs a un flot maximum de 1.

\cscript{partie2pygraph}{R\'esolution probl\`eme flot max}

\subsection{L'approche programme lin\'eaire}

Fonction d'optimisation : \\\\
$\max y$\\\\
Contraintes :\\

$(\sum_{j=1}^{m}x_{ij}.u_{ij})-y > 0, \forall i \in [0,n]$
$\prod_{i=1}^{n}\sum_{j=1}^{m}x_{ij} = 1$\\
$\prod_{j=1}^{n}\sum_{i=1}^{m}x_{ij} = 1$\\
$x_{ij}\in\{0,1\}, \forall i,j\in[0,n]$\\
$u_{ij}\in[0,M]$

\cscript{P1}{Programme lin\'eaire P1 avec pygurobi}

\subsection{Essais num\'eriques}

\begin{tabular}{|*{4}{c|}}
    \hline
     n  & P1  & Graphes\\
    \hline
     10  & 0.01  & 0.01\\
    \hline
     50  & 0.14  & 2\\
    \hline
     100  & 1.48  & 35\\
    \hline
\end{tabular}

On constate que, d'apr\`es les r\'esultats, Les Calculs avec l'algorithme de flot max sont beaucoup plus lent.
La faute a l'\'etape de construction de graphe qui devient laborieuse lorsque n est grand.

Le programme lin\'eaire P0 permet d'avoir un meilleur maximum
mais en contrepartie peut avoir un minimum tr\`es bas. 
Tandis que le programme lin\'eaire P1 permet d'avoir une moyenne 
sup\'erieur et un meilleur minimum mais en contrepartie r\'eduit le maximum.

\subsection{Approche un peu moins \'egalitariste}

Fonction d'optimisation : \\\\
$\max (y+\epsilon\sum_{i=1}^{n}\sum_{j=1}^{m}x_{ij}.u_{ij})$\\\\
Contraintes :\\

$(\sum_{j=1}^{m}x_{ij}.u_{ij})-y > 0, \forall i \in [0,n]$
$\prod_{i=1}^{n}\sum_{j=1}^{m}x_{ij} = 1$\\
$\prod_{j=1}^{n}\sum_{i=1}^{m}x_{ij} = 1$\\
$x_{ij}\in\{0,1\}, \forall i,j\in[0,n]$\\
$u_{ij}\in[0,M]$

\cscript{P2}{Programme lin\'eaire P2 avec pygurobi}


%----------------------------------------------------------------------------------------
%	Approche \'egalitariste en regrets
%----------------------------------------------------------------------------------------

\newpage
\section{Approche \'egalitariste en regrets}

\subsection{Regrets en programme lin\'eaire}

Fonction d'optimisation : \\\\
$\max y$\\\\
Contraintes :\\

$(\sum_{j=1}^{m}x_{ij}.u_{ij})-y > 0, \forall i \in [0,n]$

$\prod_{i=1}^{n}\sum_{j=1}^{m}x_{ij} = 1$\\
$\prod_{j=1}^{n}\sum_{i=1}^{m}x_{ij} = 1$\\
$x_{ij}\in\{0,1\}, \forall i,j\in[0,n]$\\
$u_{ij}\in[0,M]$

\cscript{P3}{Programme lin\'eaire P3 avec pygurobi}

\subsection{Mod\'elisation du probl\`eme sous forme de flot max}

\cscript{partie3pygraph}{R\'esolution probl\`eme flot max avec regrets (algo hongrois)}

\subsection{R\'esultats}

\begin{tabular}{|*{4}{c|}}
    \hline
     n  & P1  & Graphes\\
    \hline
     10  & 0.01  & 0.01\\
    \hline
     50  & 0.36  & 1\\
    \hline
     100  & 1.76  & 6\\
    \hline
\end{tabular}

M\^eme remarque que pour la r\'esolution par flot max \'egalitariste

%----------------------------------------------------------------------------------------
%	Extension \`a l'affectation multiple
%----------------------------------------------------------------------------------------

\newpage
\section{Extension \`a l'affectation multiple}

P0 ne peut pas s'adapter au cas générale car ses seuls contraintes sont celle du cas spécifique.
P1, P2 et P3 peuvent s'adapter au cas générale en retirent les contraintes sur le nombre d'objet qu'un agent peux avoir et en ajoutant chaque objet en plusieurs exemplaires comme de nouveaux objet.
De même pour l'algorithme de graphe égalitariste et celui des regrets dans le quelle on met le flot max de la source vers les agent à m.

\end{document}
