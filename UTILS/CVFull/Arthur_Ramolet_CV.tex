% Exemple de CV utilisant la classe moderncv
% Style classic en bleu
% Article complet : http://blog.madrzejewski.com/creer-cv-elegant-latex-moderncv/
\documentclass[11pt,a4paper]{moderncv}

\moderncvtheme[purple]{classic}                
\usepackage[utf8]{inputenc}
\usepackage[top=2cm, bottom=1cm, left=2cm, right=2cm]{geometry}

% Largeur de la colonne pour les dates
\setlength{\hintscolumnwidth}{2.5cm}

\firstname{Arthur}
\familyname{Ramolet}
\title{Développeur Unity C\# et Généraliste}              
\address{55 rue notre dame de Nazareth}{75003 Paris}    
\email{arthur.ramolet@hotmail.fr}                      
\mobile{06 59 57 49 41} 
\photo[0pt][0pt]{picture}

\begin{document}
\maketitle
\section{Expérience}
\cventry{Janvier 2020\\à Février 2020}{IT Manager à Kylotonn}{}{}{}{Gestion, maintenance et amélioration du parc informatique chez Kylotonn Racing
\begin{itemize}
\item  {Amélioration et automatisation du processus d'installation des machines.}
\item {Développement d'outils pour le monitoring du parc informatique}
\item {Maintenance et dépannage des machines .}
\end{itemize}}

\cventry{Janvier 2018\\à Janvier 2019}{Ingénieur de recherche au SGRL à l'université Champollion}{}{}{}{Développement d'un jeu sérieux (Clone) pour la formation du personnel infirmier à l'organisation de l'emploi du temps.
\begin{itemize}
\item  Conception, développement, test et expérimentation du jeu suivant les données fournies par les experts métiers et les experts pédagogiques (Unity C\#, Javascript, XML).
\item {Corrections et maintenance sur d'autres projets de l'équipe : Mecagenius et  EASYRMM }
\item {Développement d'un prototype de jeu type flappy bird contrôlé par un vélo d'appartement connecté (Unity, C\#, ANT+).}
\item {Tutorat des étudiants du Master AMINJ sur la conception d'une application en réalité augmentée pour la promotion du patrimoine (Unity C\#,Vuforia).}
\item {Formation des étudiants du Master AMINJ à l'utilisation d'Unity et d'Articy.}
\end{itemize}}

\cventry{Février 2017\\à Août 2017}{Stage recherche au LIP6, équipe MOCAH}{}{}{}{Du formalisme Entité - Composant - Système à la modélisation d’un Serious Game par un réseau de Petri.
\begin{itemize}
\item  Développement de la visite virtuelle de l’IUT de Chambéry (Unity, C\#, HTC-Vive). 
\item { Réalisation d'un outil permettant d'étiqueter des mécaniques de jeu afin de générer un réseau de pétri permettant l'analyse du suivi du joueur.}
\end{itemize}}

\cventry{Juin 2013\\à  Août 2013}{Stage consultant systèmes d’informations chez Synergie Associés}{}{}{}{Refonte d'une partie du système d'information de la mairie de Pontault-Combault. 
\begin{itemize}%
\item Automatisation de la procédure d'inscription des élèves de la commune (Web Service, Talend, PHP, SQL).
\item Simplification du système d'inscription des élèves aux activités périscolaires.
\item Améliorations des outils du portail interne de la mairie (PHP, PHPExcel, SQL).
\end{itemize}}

\section{Compétences}
\cvitem{Langages}{C\#, PHP, C/C++, SQL, HTML, CSS, Java, Python, Prolog, NetLogo, Javascript, AS3, Ruby, Delphy, VB6.}
\cvitem{Outils}{Apprentissage probabiliste, Décision automatique, Traitement du signal, Systèmes Multi-Agents , Optimisation linéaire, Complexité Algorithmique.}
\cvitem{Analyse}{UML, IHM, UX, Design Patterns.}
\cvitem{Réseaux}{Apache, DHCP, DNS, FTP, SSH, Samba, NFS.}
\cvitem{Logiciels}{Libre Office, Unity, RPGMaker, The Gimp, Blender, Talend.}
\cvlanguage{Anglais}{lu, écrit, parlé.}{}

\section{Formation}
\cventry{2014 -- 2017}{Master ANDROIDE}{Université Paris Pierre et Marie Curie}{}{}{Spécialité Agents distribués, robotique, recherche opérationnelle, interaction, décision.}
\cventry{2010 -- 2014}{Licence Sciences pour l'ingénieur / Licence Professionnelle}{Université du Maine - Le Mans}{}{}{Spécialité Informatique / Spécialité Services Informatiques et Réseaux Industriels.}

\newpage
\section{Projets personnels}

\cventry{Novembre 2019\\ à Aujourd'hui}{Création d'un jeu smartphone hypercasual}{https://hippunk.itch.io/bubblesort}{}{}{
\begin{itemize}
\item {Prototypage et implémentation des mécaniques.} 
\item {Playtests et collecte des retours des joueurs pour l'amélioration des mécaniques et modifications de gameplay.}
\end{itemize}}

\cventry{Avril 2018}{Participation à la Ludum Dare 41 en tant que développeur principal et game designer sur le jeu Axe and Roses.}{https://ldjam.com/events/ludum-dare/41/axe-and-roses-a-barbarian-date-simulator}{}{}{
\begin{itemize}
\item  Design et implémentation du moteur de dialogue et son outil d'intégration sous forme d'arbre de dialogues. 
\item {Organisation et gestion des programmeurs sur le développement des différentes briques de jeu.}
\item {Relai et organisation entre les équipes dialogues graphismes et son avec les programmeurs pour l'intégration du contenu.}
\end{itemize}}

\cventry{Decembre 2017}{Participation à la Ludum Dare 40 en tant que développeur sur le jeu \textit{Didgeribug}}{https://ldjam.com/events/ludum-dare/40/didgeribug}{}{}{
\begin{itemize}
\item  Design des mécaniques suivant les 3C conjointement avec le game designer. 
\item {Organisation et gestion des programmeurs sur le développement et l'intégration des éléments de gameplay.}
\item  Intégration des ressources graphiques et conception de l'IHM.
\end{itemize}}

\section{Projets universitaires}

\cventry{Novembre à Décembre 2016}
{Développement d'un Serious Game pour l'apprentisage du fonctionnement des défenses immunitaires dans l'organisme en Unity C\#.} { https://github.com/hippunk/ISGProjet}{}{}{
\begin{itemize}
\item Choix et modélisation de mécaniques de jeu correspondant aux connaissances à enseigner.
\item Implémentation des briques de gameplay et intégration des ressources graphiques choisies suivant le thème du jeu.
\end{itemize}
}
\cventry{Septembre à Décembre 2016}
{Utilisation d'un environnement de simulation pour expérimenter différentes stratégies de coordination de drones pour la défense en NetLogo }{ https://github.com/hippunk/Convoi}{}{}{
\begin{itemize}
\item Familiarisation avec l'environnement de simulation et le langage NetLogo
\item Développement des mécanismes et règles caractérisant les drones et agents ennemis suivant le modèle \textit{belief–desire–intention}
\item Observation et analyse des différentes stratégies implémentées
\end{itemize}}

\cventry{Décembre 2016 à Janvier 2017}{Développement d'un environnement de simulation en C++/SDL pour la reproduction de résultat de recherche sur les comportements proies/prédateurs}{ https://gitlab.com/phlf/IAR\_project   }{}{}{
\begin{itemize}
\item Modélisation et implémentation de l'architecture logicielle et la perception des agents
\item {Suivi et apprentissage des règles de codage du développeur principal et auto-formation au standard C++11.}
\end{itemize}}

\section{Centres d'intérêt}
\cvitem{Jeux vidéo}{Jeux Indépendants, Intérêt pour la conception (Notions de Game et Level design). Création amateur de jeux.}
\cvitem{Jeux de société}{Go, Jeux de plateaux, Jeux de rôles, Échecs.}
\cvitem{Hardware}{Intégrateur amateur, bonne connaissance des nouvelles technologies.}
\cvitem{Sport}{Natation et escalade en amateur.}
\cvitem{Musique}{Guitare}

\end{document}

