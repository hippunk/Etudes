% Exemple de CV utilisant la classe moderncv
% Style classic en bleu
% Article complet : http://blog.madrzejewski.com/creer-cv-elegant-latex-moderncv/
\documentclass[11pt,a4paper]{moderncv}
\moderncvtheme[purple]{classic}                
\usepackage[utf8]{inputenc}
\usepackage[top=1.1cm, bottom=1.1cm, left=2cm, right=2cm]{geometry}
% Largeur de la colonne pour les dates
\setlength{\hintscolumnwidth}{2.5cm}

\firstname{Arthur}
\familyname{Ramolet}
\title{Etudiant en Informatique}              
\address{55 rue notre dame de Nazareth}{75003 Paris}    
\email{arthur.ramolet@hotmail.fr}                      
\mobile{06 59 57 49 41} 
\photo[60pt][0pt]{picture}


\begin{document}
\maketitle

\section{Formations}
\cventry{2014 -- 2015}{Master Informatique}{Université Paris 6}{}{}{Spécialité Agents distribués, robotique, recherche opérationnelle, interaction, décision}
\cventry{2012 -- 2013}{Licence Professionnelle}{Lycée Gabriel-Touchard - Le Mans}{}{}{Spécialité Services Informatiques et Réseaux Industriels}
\cventry{2010 -- 2014}{Licence Sciences pour l'ingénieur }{Université du Maine - Le Mans}{}{}{Spécialité Informatique}
\cventry{2009 -- 2010}{Baccalauréat}{Lycée Gabriel-Touchard - Le Mans}{}{}{Série technologique, spécialité génie mécanique}

\section{Compétences}
\cvitem{Langages}{C,Java, PHP, HTML, CSS, Python, AS3, Ruby, Prolog, C++, Delphi, VB6.}
\cvitem{Outils}{Apprentissage probabiliste, Décision automatique, Optimisation linéaire, Théorie Complexité, Méthodes heuristiques.}
\cvitem{Analyse}{UML (Orienté Composants), Design d'IHM, Design Patterns.}
\cvitem{Base de données}{Oracle, PostGreSQL, MySQL, PL/SQL, SQLite.}
\cvitem{Systèmes}{Windows (XP, Seven, 8.1), Linux (Arch, CentOS, Ubuntu, Fedora).}
\cvitem{Administration}{DHCP, DNS, FTP, SSH, Samba, NFS.}
\cvitem{Réseaux}{Protocole IP, VLAN, Matériel CISCO.}
\cvitem{Logiciels}{Libre Office, The Gimp, Blender, Talend.}
\cvlanguage{Anglais}{lu, écrit, parlé}{}

\section{Expérience}
\cventry{Juin 2013\\à  Août 2013}{Stage consultant systèmes d'informations}{}{}{}{Refonte d'une partie du système d'information de la mairie de Pontault-Combault. 
\begin{itemize}%
\item Automatisation de la procédure d'inscription des élèves de la commune (Web Service, Talend, PHP, SQL).
\item Simplification du système d'inscriptions des élèves aux activités périscolaires (Publipostage, SQL).
\item Améliorations diverses des outils du portail interne de la mairie (PHP, PHPExcel, SQL).
\end{itemize}}

\cventry{Février 2013}{1er prix des 24h du code de l'ENSIM}{}{}{}{Réalisation d'un petit RPG old-school. (Nom équipe : PandaCoders). 
\begin{itemize}%
\item Fonctionalitées gra : Déplacements, Génération aléatoire de maps cohérentes.
\item ns le code : Gestion de l'inventaire, Système de combats.
\item Code source : https://github.com/hippunk/24hCodeRPG/.
\item Lien du concours : http://www.les24hducode.fr/2014/Resultats.php.
\end{itemize}}

\section{Centres d'intérêt}
\cvitem{Jeux vidéo}{Jeux Indépendants, Intérêt pour la conception (Notions de Game et Level design).}
\cvitem{Jeux de sociétés}{Go, Jeux de plateaux, Jeux de rôles, Échecs, Shogi.}
\cvitem{Hardware}{Connaissance des composants, Devis.}

\end{document}

