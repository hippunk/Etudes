%%%%%%%%%%%%%%%%%
% This is an sample CV template created using altacv.cls
% (v1.1.4, 27 July 2018) written by LianTze Lim (liantze@gmail.com). Now compiles with pdfLaTeX, XeLaTeX and LuaLaTeX.
% 
%% It may be distributed and/or modified under the
%% conditions of the LaTeX Project Public License, either version 1.3
%% of this license or (at your option) any later version.
%% The latest version of this license is in
%%    http://www.latex-project.org/lppl.txt
%% and version 1.3 or later is part of all distributions of LaTeX
%% version 2003/12/01 or later.
%%%%%%%%%%%%%%%%

%% If you need to pass whatever options to xcolor
\PassOptionsToPackage{dvipsnames}{xcolor}

%% If you are using \orcid or academicons
%% icons, make sure you have the academicons 
%% option here, and compile with XeLaTeX
%% or LuaLaTeX.
% \documentclass[10pt,a4paper,academicons]{altacv}

%% Use the "normalphoto" option if you want a normal photo instead of cropped to a circle
% \documentclass[10pt,a4paper,normalphoto]{altacv}

\documentclass[10pt,a4paper]{altacv}
%% AltaCV uses the fontawesome and academicon fonts
%% and packages. 
%% See texdoc.net/pkg/fontawecome and http://texdoc.net/pkg/academicons for full list of symbols.
%% 
%% Compile with LuaLaTeX for best results. If you
%% want to use XeLaTeX, you may need to install
%% Academicons.ttf in your operating system's font 
%% folder.


% Change the page layout if you need to
\geometry{left=1cm,right=9cm,marginparwidth=6.8cm,marginparsep=1.2cm,top=1.25cm,bottom=1.25cm,footskip=2\baselineskip}

% Change the font if you want to.

% If using pdflatex:
\usepackage[T1]{fontenc}
\usepackage[utf8]{inputenc}
\usepackage[default]{lato}

% If using xelatex or lualatex:
% \setmainfont{Lato}

% Change the colours if you want to
\definecolor{Mulberry}{HTML}{72243D}
\definecolor{SlateGrey}{HTML}{2E2E2E}
\definecolor{LightGrey}{HTML}{666666}
\colorlet{heading}{Sepia}
\colorlet{accent}{Mulberry}
\colorlet{emphasis}{SlateGrey}
\colorlet{body}{LightGrey}

% Change the bullets for itemize and rating marker
% for \cvskill if you want to
\renewcommand{\itemmarker}{{\small\textbullet}}
\renewcommand{\ratingmarker}{\faCircle}
%% sample.bib contains your publications
\addbibresource{sample.bib}

\usepackage[colorlinks]{hyperref}

\begin{document}

\definecolor{links}{HTML}{000055} 
\hypersetup{urlcolor=links} 

\name{Arthur Ramolet}
\tagline{Développeur Unity C\# }

%\photo{2.8cm}{Globe_High}
\personalinfo{%
  % Not all of these are required!
  % You can add your own with \printinfo{symbol}{detail}
  \email{\href{mailto:arthur.ramolet@hotmail.fr}{arthur.ramolet@hotmail.fr}}
  \phone{+33 6 59 57 49 41}
  %\mailaddress{Elambilakkal(Hose),Calicut airport(Po),Kondotty, Malappuram }
  \location{Paris, France}
  \linebreak
  %\homepage{www.homepage.com}
  %\twitter{@twitterhandle}
  \linkedin{\href{https://linkedin.com/in/arthur-ramolet/}{linkedin.com/in/arthur-ramolet/}}
  \github{\href{https://github.com/hippunk}{github.com/hippunk}}
  %% You MUST add the academicons option to \documentclass, then compile with LuaLaTeX or XeLaTeX, if you want to use \orcid or other academicons commands.
%   \orcid{orcid.org/0000-0000-0000-0000}
}

%% Make the header extend all the way to the right, if you want. 
\begin{fullwidth}
\makecvheader
\end{fullwidth}

%% Depending on your tastes, you may want to make fonts of itemize environments slightly smaller
% \AtBeginEnvironment{itemize}{\small}


%% Provide the file name containing the sidebar contents as an optional parameter to \cvsection.
%% You can always just use \marginpar{...} if you do
%% not need to align the top of the contents to any
%% \cvsection title in the "main" bar.

\cvsection[page1sidebar]{Expériences}

\cvevent{Ingénieur de recherche - Développeur Unity C\#}{Université Champollion - Equipe SGRL}{Janvier 2018 -- Janvier 2019}{Albi}
\begin{itemize}
\item Développement du jeu sérieux \underline{\href{https://www.univ-jfc.fr/actu/clone-un-serious-game-dedie-lorganisation-du-travail-en-milieu-hospitalier}{Clone}} en Unity C\# pour la formation du personnel infirmier.
\item Intégration web du build Unity pour récupération et analyse des traces de jeu.
\item Prototypage Unity pour contrôles de jeu avec un vélo d'appartement.
\item Formation des étudiants du Master AMINJ à l'utilisation d'Unity.
\end{itemize}

\divider

\cvevent{Stage recherche - Développeur Unity C\#}{LIP6 - Equipe MOCAH}{Février 2017 -- Août 2017}{Paris}
\begin{itemize}
\item Développement d'un Plug-in Unity pour assister la génération de trace dans les jeux sérieux.
\item Développement de la visite de l'IUT de chambéry en réalité virtuelle.
\end{itemize}

\divider

\cvevent{Stage consultant systèmes d’informations}{Synergie Associés}{Juin 2013 -- Août 2013}{Pontault-Combault}
\begin{itemize}
\item Refonte du système d'information de la mairie de Pontault-Combault pour simplifier les inscriptions pédagogiques et prendre en charge les temps périscolaires. 
\end{itemize}

\cvsection{Projets}

\cvevent{\underline{\href{https://ntzschbtzsch.itch.io/axe-and-roses}{Axe and roses}}}{Ludum Dare 41 - Developpeur principale, Game Designer}{Avril 2018}{}{Dating simulator chez les barbares.}
\divider

\cvevent{\underline{\href{https://hippunk.github.io/platypus/index.html}{Didgeribug}}}{Ludum Dare 40 - Developpeur principal}{Decembre 2017}{}{Shoot 'em up avec des insectes.}
\divider

\cvevent{\underline{\href{https://hippunk.itch.io/immunowars}{ImmunoWars}}}{Serious Game étudiant - Développeur, Game Designer}{Novembre à Décembre 2016}{}{Jeu de stratégie sur le fonctionnement des défenses immunitaires.}

\medskip

%\cvsection{Centres d'intérets}

% Adapted from @Jake's answer from http://tex.stackexchange.com/a/82729/226
% \wheelchart{outer radius}{inner radius}{
% comma-separated list of value/text width/color/detail}
%\wheelchart{1.5cm}{0.5cm}{%
%  6/8em/accent!30/{La première part.}, 
%  4/8em/accent!8/{Un dernier pour la route.},
%  3/7em/accent!8/J'ai encore faim en fait.,
%  8/8em/accent!55/Tu en as apporté ?!,
%  2/10em/accent!10/Y'en a presque plus :(.,
%  5/6em/accent!20/La fin du fromage.
%}

\clearpage
%\cvsection[page2sidebar]{Publications}

%\nocite{*}

%\printbibliography[heading=pubtype,title={\printinfo{\faBook}{Books}},type=book]

%\divider

%\printbibliography[heading=pubtype,title={\printinfo{\faFileTextO}{Journal Articles}},type=article]

%\divider

%\printbibliography[heading=pubtype,title={\printinfo{\faGroup}{Conference Proceedings}},type=inproceedings]

%% If the NEXT page doesn't start with a \cvsection but you'd
%% still like to add a sidebar, then use this command on THIS
%% page to add it. The optional argument lets you pull up the 
%% sidebar a bit so that it looks aligned with the top of the
%% main column.
% \addnextpagesidebar[-1ex]{page3sidebar}

\end{document}
